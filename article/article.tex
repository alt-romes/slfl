\documentclass{llncs}

\usepackage{cmll}
\usepackage{mathpartir}
\usepackage{amssymb}
\usepackage{amsmath}
\usepackage{hyperref}
\usepackage{todonotes}

\newcommand{\lolli}{\multimap}
\newcommand{\tensor}{\otimes}
\newcommand{\one}{\mathbf{1}}
\newcommand{\bang}{{!}}

\newcommand{\llet}[2]{\mathsf{let}\,#1\,\mathsf{in}\,#2}


\title{Synthesis of Linear Functional Programs}
\author{Rodrigo Mesquita \and Bernardo Toninho}
\date{April 2021}
\institute{NOVA School of Science and Technology}

\begin{document}

\maketitle

\section{Introduction}

% General motivation?
Program synthesis is an automated or semi-automated process of
deriving a program (i.e.~generating code) from a high-level specification.
Specifications can take many forms (e.g.  polymorphic refinement
types~\cite{DBLP:conf/pldi/PolikarpovaKS16},
examples~\cite{DBLP:conf/popl/FrankleOWZ16}, graded linear
types~\cite{DBLP:conf/lopstr/HughesO20}, ...).  In this work we
explore type-based synthesis where specifications take the form of
linear types.\todo{O que é um linear type?}
%
The chosen specification form dictates the synthesis process, and
different ones might be approached in completely different
ways.\todo{Esta frase não parece ter grande utilidade. O que queres
  dizer aqui?}
%
Regardless of the kind of specification, program synthesizers have to
deal with two main inherent sources of complexity: searching over a
vast space of (potentially) valid programs, and interpreting user
intent.
%
Synthesis has many motivations, it can be seen as a means to
improve a programmer's productivity and/or program correctness
(i.e. through suggestion and/or autocompletion), or as a tool to
automate certain parts of the programming process (e.g. in the same
way AI might generate some boilerplate text to auxiliate a journalist,
rather than to replace), just to name a few examples.
\todo[inline]{Acho que estes 3 ``paragrafos'' deviam estar por ordens diferentes.}


% Type based synthesis
Type-based synthesis takes a type and produces a program with that
given type.  Types can be further explored to give the programmer more
expressiveness (e.g. by adding constraints??), and are called rich
types.  -- Gostava de ver com o prof aqui. [Estive a procurar e fiquei
um pouco confuso. polimorfismo é rich types?  dependent types are rich
types? aqueles constraints são os dependent types pelo que
percebi... mas n me parece correto. arithmetic types? temos de
rever. Por agora vou por apenas os tópicos que o professor escreveu]
Languages with richer type systems allow for more precise types. This
precision can statically eliminate various kinds of logical errors by
making certain invalid program states ill-typed (e.g. a ``null aware''
type system, will ensure you cannot dereference a null-pointer).
However, they can also be a burden -- the whole point of these type
systems is to ensure that ``less'' things are well-typed, and
sometimes it's hard to convince the checker.  Regarding the
challenges, type-based synthesis leverages rich types as a way of
pruning the search space, and by using types gives the user as a more
``familiar'' specification.

This work explores the process of synthetising linear functional
programs from types based in linear logic (linear types), leveraging
the Curry-Howard correspondence.  Said correspondence states that
propositions in a logic have a direct mapping to types, and well typed
programs also have a one to one correspendence to proofs of those
propositions.  As such, having a type be a proposition in linear
logic, we can relate a proof of that proposition directly to a linear
functional program — finding a proof is finding a program with that
type, so we formulate the synthesis as proof search in linear logic,
which makes to us available a lot of related technology from the proof
search literature.  Linear types differ from usual types for they
constrain resource usage in programs by \emph{statically?} limiting
the number of times certain resources can be used during their
lifetime.  They can be applied to resource-aware programming such as
concurrent programming (e.g. session types??), and to
memory-management (e.g. Rust) [Quero reescrever a frase acima, não
gosto de como falamos de session types e rust em dois exemplos
paralelos visto que não são coisas bem comparáveis. Revemos na
reunião.]
 
 % Goals
In the end, we intend to be able to do full and partial synthesis of
well-typed programs -- Full synthesis being the production of a
function (or set of) satisfying the specification; Partial synthesis
being the ``completion'' of a partial program (i.e. a function with a
\emph{hole} in it) -- Starting from a small core linear --
i.e. \emph{resource-aware} functional language, and building up to
recursive types/functions.  Eventually, possibly, enhancing the
synthesis process with type refinements, and making it interactive --
i.e. the synthetiser offers the user ``choices'' mid-process.  [Devia
estar a escrever isto eu ahah?]The evaluation will be done through
expressiveness benchmarks (i.e. can we synthesize ``X''?); time
measurements (i.e. how fast can we synthesize ``X''?'); and so on.


\begin{enumerate}
\item General motivation
  \begin{itemize}
\item 
\item Specifications appear in many forms\dots
\item Program synthesis as a means of improving program
  correctness/programmer productivity.
\item Challenges: space of valid programs, space of valid
  specifications,  etc.
\end{itemize}
\item \emph{Type-based} program synthesis
  \begin{itemize}
 \item Rich types as specifications.
\item Languages with richer type systems allow for more precise
  types. This precision can statically eliminate various kinds of
  logical errors by making certain invalid program states ill-typed
  (e.g. a ``null aware'' type system, will ensure you cannot
  dereference a null-pointer).
\item But can also be a burden -- the whole point of these type
  systems is to ensure that ``less'' things are well-typed, and
  sometimes it's hard to convince the checker\dots
\item Type-based synthesis leverages rich types as a way of pruning
  the search space and by using types as a more ``familiar''
  specification.
\end{itemize}
\item ``My'' problem: Synthesis based on linear types
  \begin{itemize}
  \item Linear types constrain resource usage in programs by \emph{statically} limiting
    the number of times certain resources can be used during its
    lifetime.
  \item Applications in resource-aware programming such as concurrent
    programming (e.g. session types) and memory-management
    (e.g. Rust).
  \item This work will study program synthesis in a functional,
    linearly-typed setting, by leveraging the propositions-as-types
    correspondence between linear logic and the (linear)
    $\lambda$-calculus.
  \item What is props-as-types? What is linear logic? Linear $\lambda$
    is ``just'' a small functional core language.
  \item Through props-as-types, synthesis can be formulated as proof
    search and so we can leverage a lot of related technology from the
    literature.
  \end{itemize}
\item Goals:
  \begin{itemize}
\item Full and partial synthesis of well-typed (in the above sense)
  programs. Full synthesis means producing a function (or set of)
  satisfying the spec. Partial synthesis means being given a partial
  program (i.e. a function with a hole in it) and ``completing it''.
\item Start from a small system, build up to recursive
  types/functions.
\item Potential to go further: type refinements; interactive
  synthesis?
 \item Evaluation through expressiveness benchmarks (can we synthesize
   ``X''?); time measurements (how fast can we synthesize ``X''?');
   etc.
  \end{itemize}
   
\end{enumerate}

\section{Background}

A type system can be formally described through a set of inference
rules that inductively define a judgment of the form $\Gamma \vdash M
: A$, stating that program expression $M$ has type $A$ according to
the \emph{typing assumptions} for variables tracked in $\Gamma$.
\dots

\todo[inline]{Daqui, para tipificar uma função, para Curry-Howard,
  para lógica linear, para proof search, etc\dots Só faz sentido falar de dedução natural no
  contexto de ``lógica'' / Curry-Howard.}

natural deduction -- as a set of rules that determines the concept of deduction for some language~\cite{prawitznd65}.
The language -- the propositions of a type system are the types themselves, the rules define how to construct types, and being able to deduct a type (i.e. constructing a derivation) proves the type is valid.
[Preciso de ajuda nesta última frase, não é isto que quero dizer mas não sei descrever de forma correta e satisfatória :)]
The natural deduction judgment is extended to include a \emph{term} (i.e. to make proofs explicit with a formal definition), and modified to include a list of hypothesis.
[Também preciso de ajuda a reescrever a última, eu bem que procuro muito mas precisava de alcançar uma mestria de natural deduction - será mais rápido o professor ajudar-me a dar nomes às coisas :)]
This extended judgment, called a typing judgment has the form:
\[
    \Gamma \vdash M : A
\]
Where the natural deduction list of hypothesis, $\Gamma$, is now called \emph{typing environment} (i.e. set of variables and their types), $M$ is still a \emph{term} but also known as ``program'', and $A$, the natural deduction proposition, is now called \emph{type}. This judgment can be read -- $M$ has type $A$ in $\Gamma$.
% As a concrete example, let us describe the typing rule for the tensor type $A \tensor B$


\section{Separate}

More about the concepts than citing concrete works (but citations are
needed!):
\begin{enumerate}
\item PL/Type systems in a ``formal'' sense (inference rules, etc)
\item Propositions as types: Proofs as programs, proof search as
  synthesis, etc.
\item Linear logic: a ``resource aware logic''
\item Proof search in LL: challenges, focusing as a ``solution''
\item Linear $\lambda$-calculus: what does it look like.
\item Type-based synthesis: ``inverting'' a type system.
\end{enumerate}

\section{Related Work}

Concrete works:
\begin{itemize}
\item Synthesis for graded types (no recursion   -- in that work, no
  obvious connection of grading with
  concurrency)~\cite{DBLP:conf/lopstr/HughesO20}

\item Type-and-example-directed
  synthesis~\cite{DBLP:conf/pldi/OseraZ15,DBLP:conf/popl/FrankleOWZ16}
  or how to turn your type system upside down.
  
\item Synthesis from (polymorphic) refinement types~\cite{DBLP:conf/pldi/PolikarpovaKS16}

Refinement types provide very precise specifications to assist the
synthesis process.

\item Synthesis of heap-manipulating
  programs~\cite{DBLP:journals/pacmpl/PolikarpovaS19}

Less related, but more evidence of this idea of turning ``checking''
systems into synthesis frameworks.

\item Resource-guided synthesis \cite{DBLP:conf/pldi/KnothWP019}

This resource-guided means something a bit different. Programs satisfy
a functional specification and a symbolic resource bound in the sense
of amortized analysis, but can provide technical insights.

\end{itemize}

\section{Goals and Work Plan}

Expand on earlier points. Start from a functional language with linear
types ($\tensor$, $\lolli$, $\oplus$, $\bang$); build on it with more
``stuff''. General techniques, drawn from proof theory via props as
types: explore focusing to tame non-determinism/search space. Partial
synthesis is still proof search! Add other techniques as we go along
(e.g. for recursion we need to constrain recursive calls).

Mention that you've already implemented a type-checker for this
(useful as a prelim. exercise but also later, for \emph{validation}).

Validation and evaluation: validation is as simple as ``does it typecheck''? can you synth? how fast?

\section{Bonus}

\[
  \begin{array}{lcl}
    A, B & ::= & A \tensor B \mid A \lolli B \mid A \with B \mid A
                 \oplus B \mid \bang A \mid \one \mid \dots\\[1ex]
    M,N & ::= & \lambda x{:}A.M\\
         & \mid & M\,N\\
         & \mid & (M \tensor N)\\
         & \mid & \llet{x\tensor y = M}{N} \\
         & \mid & x \\
         & \mid & \dots\\
    \end{array}
\]

\[
  \infer[$(\lolli\! I)$]
  {\Delta , x{:}A \vdash M : B }
  {\Delta \vdash \lambda x {:} A . M : A \lolli B}
  \quad
  \infer[$(\lolli\! E)$]
  {\Delta_1 \vdash M : A \lolli B \and \Delta_2 \vdash N : A}
  {\Delta_1, \Delta_2 \vdash M\,N : B}
\]

\[
  \infer*[left=($\tensor I$)]
  {\Delta_1 \vdash M : A \and \Delta_2 \vdash N : B}
  {\Delta_1 , \Delta_2 \vdash (M \tensor N) : A \tensor B}
  \quad
  \infer*[right=($\tensor E$)]
  {\Delta_1 \vdash M : A \tensor B \and \Delta_2 , x{:}A, y{:}B\vdash
    N : C }
  {\Delta_1 , \Delta_2\vdash \llet{x\tensor y = M}{N} : C }
\]




\bibliographystyle{splncs04}
\bibliography{references}
\end{document}
