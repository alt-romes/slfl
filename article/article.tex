\documentclass{llncs}

\usepackage[inline]{todo}
\usepackage{cmll}
\usepackage{mathpartir}

\newcommand{\lolli}{\multimap}
\newcommand{\tensor}{\otimes}
\newcommand{\one}{\mathbf{1}}
\newcommand{\bang}{{!}}

\newcommand{\llet}[2]{\mathsf{let}\,#1\,\mathsf{in}\,#2}


\title{Synthesis of Linear Functional Programs}
\author{Rodrigo Mesquita}
\date{April 2021}
\institute{NOVA School of Science and Technology}

\begin{document}

\maketitle

\section{Introduction}

This work explores the process of synthetising linear functional programs from types based in linear logic, leveraging the Curry-Howard correspondence.
Said correspondence states that propositions in a logic have a direct mapping to types, and well typed programs also have a one to one correspendence to proofs of those propositions.
As such, having a type be a proposition in linear logic, we can relate a proof of that proposition directly to a linear functional program - finding a proof is finding a program with that type, and so we formulate the synthesis as proof search in linear logic.

\begin{enumerate}
\item General motivation
  \begin{itemize}
\item Program synthesis is an automated or semi-automated process of
  deriving a program from a specification of its behavior.
\item Specifications appear in many forms\dots
\item Program synthesis as a means of improving program
  correctness/programmer productivity.
\item Challenges: space of valid programs, space of valid
  specifications,  etc.
\end{itemize}
\item \emph{Type-based} program synthesis
  \begin{itemize}
 \item Rich types as specifications.
\item Languages with richer type systems allow for more precise
  types. This precision can statically eliminate various kinds of
  logical errors by making certain invalid program states ill-typed
  (e.g. a ``null aware'' type system, will ensure you cannot
  dereference a null-pointer).
\item But can also be a burden -- the whole point of these type
  systems is to ensure that ``less'' things are well-typed, and
  sometimes its hard to convince the checker\dots
\item Type-based synthesis leverages rich types as a way of pruning
  the search space and by using types as a more ``familiar''
  specification.
\end{itemize}
\item ``My'' problem: Synthesis based on linear types
  \begin{itemize}
  \item Linear types constrain resource usage in programs by \emph{statically} limiting
    the number of times certain resources can be used during its
    lifetime.
  \item Applications in resource-aware programming such as concurrent
    programming (e.g. session types) and memory-management
    (e.g. Rust).
  \item This work will study program synthesis in a functional,
    linearly-typed setting, by leveraging the propositions-as-types
    correspondence between linear logic and the (linear)
    $\lambda$-calculus.
  \item What is props-as-types? What is linear logic? Linear $\lambda$
    is ``just'' a small functional core language.
  \item Through props-as-types, synthesis can be formulated as proof
    search and so we can leverage a lot of related technology from the
    literature.
  \end{itemize}
\item Goals:
  \begin{itemize}
\item Full and partial synthesis of well-typed (in the above sense)
  programs. Full synthesis means producing a function (or set of)
  satisfying the spec. Partial synthesis means being given a partial
  program (i.e. a function with a hole in it) and ``completing it''.
\item Start from a small system, build up to recursive
  types/functions.
\item Potential to go further: type refinements; interactive
  synthesis?
 \item Evaluation through expressiveness benchmarks (can we synthesize
   ``X''?); time measurements (how fast can we synthesize ``X''?');
   etc.
  \end{itemize}
   
\end{enumerate}

\section{Background}

More about the concepts than citing concrete works (but citations are
needed!):
\begin{enumerate}
\item PL/Type systems in a ``formal'' sense (inference rules, etc)
\item Propositions as types: Proofs as programs, proof search as
  synthesis, etc.
\item Linear logic: a ``resource aware logic''
\item Proof search in LL: challenges, focusing as a ``solution''
\item Linear $\lambda$-calculus: what does it look like.
\item Type-based synthesis: ``inverting'' a type system.
\end{enumerate}

\section{Related Work}

Concrete works:
\begin{itemize}
\item Synthesis for graded types (no recursion   -- in that work, no
  obvious connection of grading with
  concurrency)~\cite{DBLP:conf/lopstr/HughesO20}

\item Type-and-example-directed
  synthesis~\cite{DBLP:conf/pldi/OseraZ15,DBLP:conf/popl/FrankleOWZ16}
  or how to turn your type system upside down.
  
\item Synthesis from (polymorphic) refinement types~\cite{DBLP:conf/pldi/PolikarpovaKS16}

Refinement types provide very precise specifications to assist the
synthesis process.

\item Synthesis of heap-manipulating
  programs~\cite{DBLP:journals/pacmpl/PolikarpovaS19}

Less related, but more evidence of this idea of turning ``checking''
systems into synthesis frameworks.

\item Resource-guided synthesis \cite{DBLP:conf/pldi/KnothWP019}

This resource-guided means something a bit different. Programs satisfy
a functional specification and a symbolic resource bound in the sense
of amortized analysis, but can provide technical insights.

\end{itemize}

\section{Goals and Work Plan}

Expand on earlier points. Start from a functional language with linear
types ($\tensor$, $\lolli$, $\oplus$, $\bang$); build on it with more
``stuff''. General techniques, drawn from proof theory via props as
types: explore focusing to tame non-determinism/search space. Partial
synthesis is still proof search! Add other techniques as we go along
(e.g. for recursion we need to constrain recursive calls).

Mention that you've already implemented a type-checker for this
(useful as a prelim. exercise but also later, for \emph{validation}).

Validation and evaluation: validation is as simple as ``does it typecheck''? can you synth? how fast?

\section{Bonus}

\[
  \begin{array}{lcl}
    A, B & ::= & A \tensor B \mid A \lolli B \mid A \with B \mid A
                 \oplus B \mid \bang A \mid \one \mid \dots\\[1ex]
    M,N & ::= & \lambda x{:}A.M\\
         & \mid & M\,N\\
         & \mid & (M \tensor N)\\
         & \mid & \llet{x\tensor y = M}{N} \\
         & \mid & x \\
         & \mid & \dots\\
    \end{array}
\]

\[
  \infer[$(\lolli\! I)$]
  {\Delta , x{:}A \vdash M : B }
  {\Delta \vdash \lambda x {:} A . M : A \lolli B}
  \quad
  \infer[$(\lolli\! E)$]
  {\Delta_1 \vdash M : A \lolli B \and \Delta_2 \vdash N : A}
  {\Delta_1, \Delta_2 \vdash M\,N : B}
\]
\[
  \infer*[left=($\tensor I$)]
  {\Delta_1 \vdash M : A \and \Delta_2 \vdash N : B}
  {\Delta_1 , \Delta_2 \vdash (M \tensor N) : A \tensor B}
  \quad
  \infer*[right=($\tensor E$)]
  {\Delta_1 \vdash M : A \tensor B \and \Delta_2 , x{:}A, y{:}B\vdash
    N : C }
  {\Delta_1 , \Delta_2\vdash \llet{x\tensor y = M}{N} : C }
\]




\bibliographystyle{splncs04}
\bibliography{references}
\end{document}
